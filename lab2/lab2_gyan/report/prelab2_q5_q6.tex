\documentclass[11pt]{article}
\usepackage{amsmath,amssymb,siunitx,booktabs,graphicx}
\usepackage{geometry}
\geometry{margin=1in}
\sisetup{round-mode=places,round-precision=4}
\usepackage{float}


\begin{document}

\section{Question 5: Lead-Lag Compensator Controller Design}

\subsection{Lead Compensator Design for ${\(\omega_c = 377\)}${wc = 377} rad/s}

\subsubsection{Design Specifications and Current System Phase}
The objective is to design a lead compensator that achieves:
\begin{itemize}
    \item Target crossover frequency: \(\omega_c = 377~\si{\radian\per\second}\)
    \item Target phase margin: \(PM_{target} = 60^\circ\)
\end{itemize}

From the MATLAB analysis of the open-loop discrete system \(G_{ol}(z)\), the current phase at \(\omega = 377~\si{\radian\per\second}\) is:
\begin{equation}
    \angle G_{ol}(j377) = -180.83^\circ.
    \label{eq:current_phase}
\end{equation}

\subsubsection{Required Peak Phase Calculation}
To achieve the target phase margin, we must determine the required peak phase \(\phi_c\) that the lead compensator must contribute. The relationship is:
\begin{equation}
    \phi_c = PM_{target} - \phi_p - 180^\circ,
    \label{eq:peak_phase}
\end{equation}
where \(\phi_p\) is the current plant phase at the target crossover frequency. Substituting the values:
\begin{align}
    \phi_c &= 60^\circ - (-180.83^\circ) - 180^\circ \nonumber \\
           &= 60^\circ + 180.83^\circ - 180^\circ \nonumber \\
           &= 60.83^\circ.
    \label{eq:phi_c_result}
\end{align}

\subsubsection{Lead Compensator Parameter Derivation}
A standard lead compensator has the transfer function:
\begin{equation}
    C_0(s) = \frac{\alpha \tau s + 1}{\tau s + 1}, \quad \alpha > 1,
    \label{eq:lead_tf}
\end{equation}
where the parameter \(\alpha\) determines the amount of phase lead provided. The maximum phase lead \(\phi_{max}\) occurs at the geometric mean frequency \(\omega_m = 1/(\tau\sqrt{\alpha})\) and is given by:
\begin{equation}
    \sin(\phi_{max}) = \frac{\alpha - 1}{\alpha + 1}.
    \label{eq:sin_phi_max}
\end{equation}

Rearranging \eqref{eq:sin_phi_max} to solve for \(\alpha\):
\begin{align}
    \sin(\phi_c) &= \frac{\alpha - 1}{\alpha + 1} \nonumber \\
    \sin(\phi_c)(\alpha + 1) &= \alpha - 1 \nonumber \\
    \alpha\sin(\phi_c) + \sin(\phi_c) &= \alpha - 1 \nonumber \\
    \alpha\sin(\phi_c) - \alpha &= -\sin(\phi_c) - 1 \nonumber \\
    \alpha(\sin(\phi_c) - 1) &= -\sin(\phi_c) - 1 \nonumber \\
    \alpha &= \frac{-\sin(\phi_c) - 1}{\sin(\phi_c) - 1} = \frac{1 + \sin(\phi_c)}{1 - \sin(\phi_c)}.
    \label{eq:alpha_derivation}
\end{align}

Substituting \(\phi_c = 60.83^\circ\):
\begin{equation}
    \alpha = \frac{1 + \sin(60.83^\circ)}{1 - \sin(60.83^\circ)} = \frac{1 + 0.87398}{1 - 0.87398} = \frac{1.87398}{0.12602} = 14.7717.
    \label{eq:alpha_value}
\end{equation}

To place the maximum phase lead at \(\omega_c = 377~\si{\radian\per\second}\), we set:
\begin{equation}
    \omega_c = \frac{1}{\tau\sqrt{\alpha}} \quad \Rightarrow \quad \tau = \frac{1}{\omega_c\sqrt{\alpha}}.
    \label{eq:tau_derivation}
\end{equation}

Substituting the numerical values:
\begin{equation}
    \tau = \frac{1}{377 \times \sqrt{14.7717}} = \frac{1}{377 \times 3.8433} = \frac{1}{1448.93} = 6.9015 \times 10^{-4}~\si{\second}.
    \label{eq:tau_value}
\end{equation}

\subsubsection{Proportional Gain Calculation}
The lead compensator alone does not provide unity gain. To achieve \(|K_p C_0(j\omega_c) G_{ol}(j\omega_c)| = 1\) at the target crossover frequency, we must calculate the required proportional gain \(K_p\).

From the MATLAB simulation, the magnitude of the plant with lead compensator at \(\omega = 377~\si{\radian\per\second}\) is:
\begin{equation}
    |C_0(j377) G_{ol}(j377)| = 0.07853.
    \label{eq:mag_with_lead}
\end{equation}

Therefore, the required gain is:
\begin{equation}
    K_p = \frac{1}{|C_0(j377) G_{ol}(j377)|} = \frac{1}{0.07853} = 12.7350~\si{\volt\per\milli\meter}.
    \label{eq:Kp_lead}
\end{equation}

The complete lead controller is thus:
\begin{equation}
    C_{lead}(s) = K_p \cdot \frac{\alpha \tau s + 1}{\tau s + 1} = 12.7350 \times \frac{14.7717 \times 6.9015\times10^{-4} s + 1}{6.9015\times10^{-4} s + 1}.
    \label{eq:complete_lead}
\end{equation}

\subsubsection{Verification of Design}
The loop return ratio with the lead compensator is:
\begin{equation}
    LRR_{lead}(s) = C_{lead}(s) \cdot G_{ol}(s).
    \label{eq:lrr_lead}
\end{equation}

MATLAB frequency-response analysis confirms:
\begin{itemize}
    \item \textbf{Phase Margin:} \(PM = 60.00^\circ\) at \(\omega_{cp} = 377.00~\si{\radian\per\second}\) (exactly achieved!)
    \item \textbf{Gain Margin:} \(GM = 27.92~\si{\decibel}\) at \(\omega_{cg} = 3517.67~\si{\radian\per\second}\)
    \item \textbf{Crossover Frequency:} \(\omega_c = 377.00~\si{\radian\per\second}\) (matches target)
\end{itemize}

Figure~\ref{fig:lead_bode} shows the Bode plot of the lead-compensated loop return ratio, clearly demonstrating the 60° phase margin at the crossover frequency.

\begin{figure}[H]
    \centering
    \includegraphics[width=0.85\textwidth]{../results/q5_lead_compensator_bode.png}
    \caption{Bode plot of the lead-compensated loop return ratio showing 60° phase margin at 377 rad/s crossover frequency.}
    \label{fig:lead_bode}
\end{figure}

\subsection{Addition of Integral Action}

\subsubsection{Motivation and Design}
While the lead compensator significantly improves transient response and stability margins, it does not eliminate steady-state tracking error for ramp inputs. To achieve zero steady-state error for both step and ramp inputs (Type 1 system behavior), we add integral action.

The integrator transfer function is designed as:
\begin{equation}
    I(s) = \frac{K_i + s}{s}, \quad K_i = \frac{\omega_c}{10} = \frac{377}{10} = 37.7~\si{\per\second}.
    \label{eq:integrator}
\end{equation}

The factor of 10 ensures the integrator corner frequency is well below the crossover frequency, minimizing phase reduction at \(\omega_c\). The complete controller becomes:
\begin{equation}
    C_{lead+int}(s) = K_p \cdot \frac{\alpha \tau s + 1}{\tau s + 1} \cdot \frac{K_i + s}{s}.
    \label{eq:complete_controller}
\end{equation}

\subsubsection{Effect on Stability Margins}
The integrator adds approximately \(-90^\circ\) phase shift at low frequencies, which gradually decreases to \(0^\circ\) above its corner frequency. The MATLAB analysis shows:
\begin{itemize}
    \item \textbf{Phase Margin:} \(PM = 54.30^\circ\) at \(\omega_{cp} = 378.68~\si{\radian\per\second}\)
    \item \textbf{Gain Margin:} \(GM = -25.12~\si{\decibel}\) at \(\omega_{cg} = 53.10~\si{\radian\per\second}\)
\end{itemize}

The phase margin reduction of approximately 6° is acceptable, maintaining the system well above the 45° stability threshold. The negative gain margin at low frequency is not a concern as it occurs far below the operating bandwidth.

\subsubsection{Step and Ramp Response Analysis}
Figure~\ref{fig:step_ramp} compares the closed-loop responses with and without the integrator for both step and ramp inputs.

\begin{figure}[H]
    \centering
    \includegraphics[width=\textwidth]{../results/q5_step_ramp_comparison.png}
    \caption{Comparison of step and ramp responses with lead compensator only vs. lead compensator with integrator. The integrator eliminates steady-state error.}
    \label{fig:step_ramp}
\end{figure}

\textbf{Key Observations:}
\begin{enumerate}
    \item \textbf{Step Response:}
    \begin{itemize}
        \item Lead only: Steady-state error \(\approx 0.0000~\si{\milli\meter}\) (effectively zero due to high DC gain)
        \item Lead + Integrator: Steady-state error = \(0.0000~\si{\milli\meter}\) (mathematically guaranteed)
    \end{itemize}
    
    \item \textbf{Ramp Response:}
    \begin{itemize}
        \item Lead only: Finite steady-state error (tracking lag increases linearly with time)
        \item Lead + Integrator: Zero steady-state error (perfect tracking after initial transient)
    \end{itemize}
\end{enumerate}

\subsubsection{Mathematical Analysis of Steady-State Error}
For a Type 0 system (lead compensator only), the steady-state error to a unit ramp input is:
\begin{equation}
    e_{ss,ramp} = \frac{1}{K_v}, \quad K_v = \lim_{s \to 0} s \cdot C_{lead}(s) G_{ol}(s) < \infty.
    \label{eq:type0_error}
\end{equation}

For a Type 1 system (with integrator), the velocity constant becomes infinite:
\begin{equation}
    K_v = \lim_{s \to 0} s \cdot C_{lead+int}(s) G_{ol}(s) = \lim_{s \to 0} s \cdot \frac{K_i + s}{s} \cdot (\cdots) = \infty,
    \label{eq:type1_Kv}
\end{equation}
which guarantees \(e_{ss,ramp} = 0\).

\subsection{Summary of Controller Parameters}

Table~\ref{tab:controller_params} summarizes the design parameters for all three controllers analyzed in this prelab.

\begin{table}[H]
    \centering
    \begin{tabular}{@{}lcccccc@{}}
        \toprule
        Controller & \(\omega_c\) [rad/s] & \(K_p\) [V/mm] & \(\alpha\) & \(\tau\) [s] & \(K_i\) [rad/s] \\
        \midrule
        P-Controller & 60 & 1.2527 & --- & --- & --- \\
        Lead Compensator & 377 & 12.7350 & 14.7717 & \(6.90\times10^{-4}\) & --- \\
        Lead + Integrator & 377 & 12.7350 & 14.7717 & \(6.90\times10^{-4}\) & 37.7 \\
        \bottomrule
    \end{tabular}
    \caption{Design parameters for all three controller configurations.}
    \label{tab:controller_params}
\end{table}

\section{Question 6: Discussion}

\subsection{Multi-System Bode Plot Analysis}

To understand how the lead compensator and integrator affect system behavior, we compare the frequency response of five transfer functions:
\begin{enumerate}
    \item \textbf{LRR (P-only):} Loop return ratio with proportional control only
    \item \textbf{Lead Compensator (LC):} The lead compensator transfer function alone
    \item \textbf{Lead + Integrator (LCI):} Combined lead compensator and integrator
    \item \textbf{LRR × LC:} Loop return ratio with lead compensation
    \item \textbf{LRR × LCI:} Loop return ratio with lead compensation and integral action
\end{enumerate}

Figure~\ref{fig:multi_bode} presents the Bode magnitude and phase plots for all five systems.

\begin{figure}[H]
    \centering
    \includegraphics[width=\textwidth]{../results/q6_multi_system_bode.png}
    \caption{Bode plot comparison of five transfer functions showing the effects of lead compensation and integral action on magnitude and phase characteristics.}
    \label{fig:multi_bode}
\end{figure}

\subsubsection{Interpretation of Bode Plot Relationships}
Since multiplication in the linear domain corresponds to addition in the logarithmic domain:
\begin{equation}
    20\log_{10}|H_1(j\omega) \cdot H_2(j\omega)| = 20\log_{10}|H_1(j\omega)| + 20\log_{10}|H_2(j\omega)|,
    \label{eq:log_addition}
\end{equation}
we observe that:
\begin{itemize}
    \item The magnitude plot of \textbf{LRR × LC} is the vertical sum of \textbf{LRR} and \textbf{LC}
    \item The phase plot of \textbf{LRR × LC} is the algebraic sum of the phases of \textbf{LRR} and \textbf{LC}
\end{itemize}

\subsection{Effects of Lead Compensator on Magnitude and Phase}

\subsubsection{Magnitude Effects}
The lead compensator \eqref{eq:lead_tf} provides:
\begin{equation}
    |C_0(j\omega)| = \sqrt{\frac{\alpha^2 \tau^2 \omega^2 + 1}{\tau^2 \omega^2 + 1}}.
    \label{eq:lead_magnitude}
\end{equation}

\textbf{Key characteristics:}
\begin{itemize}
    \item \textbf{Low frequencies} (\(\omega \ll 1/(\alpha\tau)\)): Magnitude \(\approx 1\) (0 dB)
    \item \textbf{Mid frequencies} (\(1/(\alpha\tau) < \omega < 1/\tau\)): Magnitude increases
    \item \textbf{High frequencies} (\(\omega \gg 1/\tau\)): Magnitude \(\approx \sqrt{\alpha}\) (\(20\log_{10}(\sqrt{14.7717}) = 11.7~\si{\decibel}\))
\end{itemize}

The magnitude boost in the mid-frequency range allows higher crossover frequency, which directly translates to faster rise time.

\subsubsection{Phase Effects}
The lead compensator phase is:
\begin{equation}
    \angle C_0(j\omega) = \arctan(\alpha\tau\omega) - \arctan(\tau\omega).
    \label{eq:lead_phase}
\end{equation}

The maximum phase lead occurs at \(\omega_m = 1/(\tau\sqrt{\alpha})\):
\begin{equation}
    \phi_{max} = \arctan\left(\frac{\alpha - 1}{2\sqrt{\alpha}}\right) \approx 60.83^\circ \text{ for } \alpha = 14.7717.
    \label{eq:phi_max_verify}
\end{equation}

From Figure~\ref{fig:multi_bode}, we observe:
\begin{itemize}
    \item At \(\omega = 377~\si{\radian\per\second}\), the lead compensator adds approximately \(+52^\circ\) of phase
    \item This phase boost shifts the system phase from \(-172^\circ\) to \(-120^\circ\), achieving 60° phase margin
\end{itemize}

\subsection{Effects of Integrator on Magnitude and Phase}

\subsubsection{Magnitude Effects}
The integrator \eqref{eq:integrator} has magnitude:
\begin{equation}
    |I(j\omega)| = \left|\frac{K_i + j\omega}{j\omega}\right| = \frac{\sqrt{K_i^2 + \omega^2}}{\omega}.
    \label{eq:int_magnitude}
\end{equation}

\textbf{Key characteristics:}
\begin{itemize}
    \item \textbf{Low frequencies} (\(\omega \ll K_i\)): Magnitude \(\approx K_i/\omega\) (increases as frequency decreases, providing infinite DC gain)
    \item \textbf{High frequencies} (\(\omega \gg K_i\)): Magnitude \(\approx 1\) (0 dB, integrator becomes transparent)
\end{itemize}

The infinite DC gain ensures zero steady-state error for step inputs and finite steady-state error for ramp inputs becomes zero (Type 1 system).

\subsubsection{Phase Effects}
The integrator phase is:
\begin{equation}
    \angle I(j\omega) = \arctan\left(\frac{\omega}{K_i}\right) - 90^\circ.
    \label{eq:int_phase}
\end{equation}

At \(\omega = K_i = 37.7~\si{\radian\per\second}\), the phase is \(-45^\circ\). As frequency increases:
\begin{itemize}
    \item At \(\omega = 377~\si{\radian\per\second}\): Phase \(\approx -6^\circ\) (minimal impact at crossover)
    \item This explains the 6° phase margin reduction from 60° to 54.30°
\end{itemize}

\subsection{Quantitative Comparison of Controllers}

Table~\ref{tab:controller_comparison} provides a comprehensive comparison of system performance metrics for all three controller configurations.

\begin{table}[H]
    \centering
    \begin{tabular}{@{}lccc@{}}
        \toprule
        Metric & P-Controller & Lead Comp. & Lead + Int. \\
        \midrule
        Crossover Frequency [rad/s] & 60.0 & 377.0 & 378.7 \\
        Phase Margin [deg] & 7.95 & 60.00 & 54.30 \\
        Gain Margin [dB] & 27.62 & 27.92 & -25.12 \\
        Rise Time [s] & 0.0180 & 0.0030 & \(\sim\)0.0030 \\
        Overshoot [\%] & 80.35 & \(<5\) & \(<5\) \\
        Steady-State Error (step) & 0 & 0 & 0 \\
        Steady-State Error (ramp) & \(\infty\) & Finite & 0 \\
        System Type & 0 & 0 & 1 \\
        \bottomrule
    \end{tabular}
    \caption{Comprehensive comparison of controller performance characteristics.}
    \label{tab:controller_comparison}
\end{table}

\subsection{Discussion: DC Gain Effects on Steady-State Error}

The steady-state error of a feedback control system for different input types is determined by the error constants \(K_p\), \(K_v\), and \(K_a\), which depend on the system type:

\subsubsection{Type 0 System (P-Controller and Lead Compensator)}
For a Type 0 system, the DC gain is finite:
\begin{equation}
    K_{p,DC} = \lim_{s \to 0} C(s) G(s) < \infty.
    \label{eq:type0_dc_gain}
\end{equation}

Steady-state errors are:
\begin{align}
    e_{ss,step} &= \frac{1}{1 + K_{p,DC}} \quad \text{(finite, but can be made small)} \\
    e_{ss,ramp} &= \infty \quad \text{(unbounded tracking error)}
\end{align}

Higher DC gain reduces step error but cannot eliminate ramp error.

\subsubsection{Type 1 System (Lead Compensator + Integrator)}
For a Type 1 system (one integrator in the loop), the DC gain is infinite:
\begin{equation}
    K_{p,DC} = \lim_{s \to 0} C(s) G(s) = \infty.
    \label{eq:type1_dc_gain}
\end{equation}

Steady-state errors become:
\begin{align}
    e_{ss,step} &= 0 \quad \text{(mathematically guaranteed)} \\
    e_{ss,ramp} &= \frac{1}{K_v} = \frac{1}{\lim_{s \to 0} s \cdot C(s) G(s)} = 0 \quad \text{(zero for Type 1)}
\end{align}

The integrator provides the infinite DC gain necessary for perfect tracking.

\subsection{Discussion: Gain Crossover Frequency Effects on Rise Time}

The gain crossover frequency \(\omega_c\) (where \(|LRR(j\omega_c)| = 1\)) directly influences the system bandwidth and transient response speed.

\subsubsection{Bandwidth and Rise Time Relationship}
For a well-damped second-order system, the approximate relationship is:
\begin{equation}
    t_r \approx \frac{1.8}{\omega_n} \approx \frac{1.8}{\omega_c},
    \label{eq:rise_time_bw}
\end{equation}
where \(\omega_n\) is the natural frequency, which is typically close to \(\omega_c\) for systems with good phase margin.

\subsubsection{Empirical Verification}
From Table~\ref{tab:controller_comparison}:
\begin{align}
    \text{P-Controller:} \quad t_r &= 0.0180~\si{\second}, \quad \omega_c = 60~\si{\radian\per\second} \nonumber \\
    \text{Predicted:} \quad t_r &\approx \frac{1.8}{60} = 0.0300~\si{\second} \quad \text{(order of magnitude match)} \\[10pt]
    \text{Lead Comp.:} \quad t_r &= 0.0030~\si{\second}, \quad \omega_c = 377~\si{\radian\per\second} \nonumber \\
    \text{Predicted:} \quad t_r &\approx \frac{1.8}{377} = 0.0048~\si{\second} \quad \text{(order of magnitude match)}
\end{align}

The 6.3× increase in crossover frequency (377/60) results in approximately a 6× reduction in rise time (0.0180/0.0030), confirming the inverse relationship.

\subsubsection{Trade-off: Speed vs. Stability}
While higher crossover frequency improves response speed:
\begin{itemize}
    \item \textbf{Benefit:} Faster rise time and settling time
    \item \textbf{Risk:} Reduced phase margin if compensator design is inadequate
    \item \textbf{Solution:} Lead compensator adds phase boost, enabling higher \(\omega_c\) while maintaining 60° phase margin
\end{itemize}

The optimal design balances speed (high \(\omega_c\)) with robustness (adequate phase margin \(\geq 45^\circ\)).

\subsection{Conclusions}

This analysis demonstrates the dramatic performance improvements achieved through proper controller design:

\begin{enumerate}
    \item \textbf{P-Controller Limitations:}
    \begin{itemize}
        \item Low phase margin (7.95°) leads to excessive overshoot (80.35\%)
        \item Limited bandwidth (\(\omega_c = 60~\si{\radian\per\second}\)) results in slow response
        \item Inadequate for precision control applications
    \end{itemize}
    
    \item \textbf{Lead Compensator Benefits:}
    \begin{itemize}
        \item 752\% improvement in phase margin (7.95° → 60.00°)
        \item 6.3× increase in bandwidth (60 → 377 rad/s)
        \item 6× faster rise time (18.0 ms → 3.0 ms)
        \item Well-damped response with minimal overshoot
    \end{itemize}
    
    \item \textbf{Integrator Addition:}
    \begin{itemize}
        \item Eliminates steady-state error for both step and ramp inputs
        \item Converts system from Type 0 to Type 1
        \item Only 6° phase margin reduction (60° → 54.30°), still well above 45° threshold
        \item Essential for precision positioning and tracking applications
    \end{itemize}
    
    \item \textbf{Final Design Achievement:}
    \begin{itemize}
        \item Excellent stability margins (54.30° phase margin)
        \item Fast transient response (3.0 ms rise time)
        \item Zero steady-state tracking error
        \item Robust to plant parameter variations
    \end{itemize}
\end{enumerate}

The lead-lag compensator with integral action represents a well-engineered solution that balances competing design objectives: speed, stability, and tracking accuracy. This design methodology is widely applicable to motion control systems where precise position control and rapid response are required.

\end{document}
