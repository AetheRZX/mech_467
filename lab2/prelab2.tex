\documentclass[12pt]{article}
% --- Page geometry & core packages ---
\usepackage{geometry}
\geometry{margin=1in}
\usepackage{graphicx}
\usepackage{longtable}
\usepackage{booktabs}
\usepackage{siunitx}
\usepackage{float}
\usepackage{amsmath}
\usepackage[hidelinks]{hyperref}
\usepackage{multirow}
\usepackage[table]{xcolor}
\usepackage[most]{tcolorbox}
% --- Fonts & microtypography (Times-like) ---
\usepackage{newtxtext,newtxmath}
\usepackage{microtype}
% --- In your preamble (once) ---
\usepackage{listings}
\usepackage[T1]{fontenc}
\usepackage[scaled=0.85]{beramono}
\lstdefinestyle{pycode}{
  language=Python,
  basicstyle=\ttfamily\small,
  numbers=left,
  numberstyle=\scriptsize\color{black!50},
  numbersep=8pt,
  frame=single,
  framerule=0.4pt,
  rulecolor=\color{black!20},
  showstringspaces=false,
  breaklines=true,
  columns=fullflexible,
  tabsize=4,
  literate={
    {→}{{$\to$}}1
    {←}{{$\leftarrow$}}1
    {·}{{$\cdot$}}1
    {×}{{$\times$}}1
    {–}{{-}}1
    {—}{{-}}1
    {−}{{-}}1
    {≥}{{$\ge$}}1
    {≤}{{$\le$}}1
    {°}{{$^\circ$}}1
    {Ω}{{$\Omega$}}1
    {µ}{{$\mu$}}1
  }
}
\lstset{style=pycode}
% --- Captions ---
\usepackage{caption}
\captionsetup{
  font=small,
  labelfont=bf,
  labelsep=period,
  justification=centering,
  singlelinecheck=false,
  skip=6pt
}
\captionsetup[table]{position=top}
% --- Section headings ---
\usepackage{titlesec}
\titleformat{\section}{\Large\sffamily\bfseries}{\thesection}{0.6em}{}
\titleformat{\subsection}{\large\sffamily\bfseries}{\thesubsection}{0.5em}{}
\titlespacing*{\section}{0pt}{1.2ex plus .3ex}{0.6ex}
\titlespacing*{\subsection}{0pt}{1.0ex plus .2ex}{0.4ex}
% --- Header / footer ---
\usepackage{fancyhdr}
\pagestyle{fancy}
\fancyhf{}
\newcommand{\coursenum}{MECH 467}
\newcommand{\labnum}{Prelab 2}
\newcommand{\labtitle}{Digital Control of the Ball-Screw Feed Drive}
\newcommand{\studentname}{Ryan Edric Nashota}
\newcommand{\studentid}{33800060}
\lhead{\sffamily \labnum}
\chead{\sffamily \labtitle}
\rhead{\sffamily \studentname}
\renewcommand{\headrulewidth}{0.4pt}
\cfoot{\sffamily \thepage}
\setlength{\headheight}{15pt}
\addtolength{\topmargin}{-3pt}
\usepackage{titlesec}
\titleformat{\paragraph}[block]{\normalfont\normalsize\bfseries}{}{}{}
\titlespacing*{\paragraph}{0pt}{0.8ex}{0.6ex}
% --- Cleveref ---
\usepackage[nameinlink,noabbrev]{cleveref}
\Crefname{figure}{Figure}{Figures}
\Crefname{table}{Table}{Tables}
\Crefname{equation}{Equation}{Equations}
\Crefname{section}{Section}{Sections}
\crefname{figure}{figure}{figures}
\crefname{table}{table}{tables}
\crefname{equation}{equation}{equations}
\crefname{section}{section}{sections}
% --- Graphics path ---
\graphicspath{{../analysis_outputs/figures/}{figures/}}
\newtcolorbox{infobox}[1][]{colback=blue!5,colframe=cyan!60!black,title=#1,boxrule=0.6pt}
% --- Unicode shortcuts ---
\usepackage{newunicodechar}
% --- siunitx custom units ---
\DeclareSIUnit{\lbf}{lbf}
\DeclareSIUnit{\inch}{in}
\DeclareSIUnit{\lbfin}{lb\,in}
% --- URL tweaks ---
\usepackage{url}
\def\UrlBreaks{\do\_\do\-}
\newcommand{\code}[1]{\texttt{\nolinkurl{#1}}}
\emergencystretch=2em
% --- Title block ---
\makeatletter
\renewcommand{\maketitle}{
  \vspace*{1ex}
  \begin{center}
    {\sffamily
      {\Large \coursenum\ --- \labnum\par}
      \vspace{0.6ex}
      {\huge \bfseries \labtitle\par}
      \vspace{0.8ex}
      {\large \studentname\enspace (ID:\ \studentid)\par}
      \vspace{0.6ex}
      {\normalsize Lab Performed:\ 12~February~2025 \\
      Report Submitted:\ 19~February~2025\par}
    }
  \end{center}
  \vspace{1.2ex}
  \thispagestyle{empty}
}
\makeatother
% --- Body spacing ---
\setlength{\parindent}{0pt}
\setlength{\parskip}{0.6em}
% --- Metadata ---
\title{\coursenum\ \labnum\\\labtitle}
\author{\studentname\\Student ID: \studentid}
\date{}
\begin{document}
\pagenumbering{gobble}
\maketitle
\clearpage
\pagenumbering{roman}
\setcounter{page}{1}
\renewcommand{\contentsname}{Table of Contents}
\tableofcontents
\clearpage
\renewcommand{\listfigurename}{Table of Figures}
\listoffigures
\clearpage
\listoftables
\clearpage
\setcounter{tocdepth}{2}
\pagenumbering{arabic}
\setcounter{page}{1}
\pagestyle{fancy}

\section{Introduction}

The open-loop model from \Cref{fig:analyzed_system} of the Project~II 
handout removes the Coulomb friction and saturation elements and collapses the amplifier, torque constant, inertia, damping, and encoder dynamics into a single continuous-time transfer function.  Using the provided parameters ($K_a=0.887~\si{A/V}$, $K_t=0.72~\si{Nm/A}$, $J_e=7\times10^{-4}~\si{kgm^2}$, $B_e=0.00612~\si{Nms/rad}$, $K_e=20/(2\pi)~\si{mm/rad}$, $K_d=1$) the open-loop gain is
\begin{figure}[H]
  
\end{figure}
 \centering
  \includegraphics[width=0.7\linewidth]{system_analyzed.png}
  \caption{Analyzed System from Project Handout}
  \label{fig:analyzed_system}
\begin{equation}
  G_\text{ol}(s) = \frac{K_a K_t K_e}{s(J_e s + B_e)}
  = \frac{2.0329}{0.0007 s^2 + 0.00612 s}.
  \label{eq:plant}
\end{equation}
The pole at the origin reflects displacement integration while the real pole at $-B_e/J_e=-8.743~\si{rad/s}$ captures the motor--ball-screw mechanical time constant.  The subsequent sections translate this model into discrete form and design digital controllers that satisfy the pre-lab deliverables from \cite{project_handout}.

\section{Prelab 1 -- Discrete Transfer Function Derivation}
\subsection{Zero-Order Hold Derivation}
The partial-fraction decomposition of \Cref{eq:plant} is
\begin{equation}
  G_\text{ol}(s) = \frac{332.166}{s} - \frac{37.993}{0.11438\,s + 1}
  = \frac{332.166}{s} - \frac{37.993}{s + 8.7429}.
  \label{eq:partial_frac}
\end{equation}
Applying the standard ZOH expressions
\begin{align}
  \mathcal{Z}\left\{\frac{1}{s}\right\} &= \frac{T z^{-1}}{1 - z^{-1}},\\
  \mathcal{Z}\left\{\frac{1}{s + a}\right\} &= \frac{(1 - e^{-aT}) z^{-1}}{1 - e^{-aT} z^{-1}},
\end{align}
with $T=0.0002~\si{s}$ yields the zero-order-hold equivalent
\begin{equation}
  G_\text{ol}(z) = \frac{332.166\,T z^{-1}}{1 - z^{-1}}
    - \frac{37.993 (1 - e^{-8.7429 T}) z^{-1}}{1 - e^{-8.7429 T} z^{-1}}.
  \label{eq:zoh_symbolic}
\end{equation}
Substituting the numerical constants and simplifying gives
\begin{equation}
  G_\text{ol}(z) = \frac{5.8048\times10^{-5} z + 5.8014\times10^{-5}}
                        {z^2 - 1.99825296 z + 0.99825296}.
  \label{eq:zoh_numeric}
\end{equation}

\subsection{Comparison with MATLAB \texttt{c2d}}
Equation~\eqref{eq:zoh_numeric} matches the discrete transfer function reported by \texttt{c2d(G\_ol,~0.0002,~'zoh')} in MATLAB to all significant digits.  The numerator and denominator coefficients from MATLAB
are shown in \Cref{fig:c2d_matlab}

\begin{figure}[H]
  \centering
  \includegraphics[width=0.7\linewidth]{c2d.png}
  \caption{Output from MATLAB c2d}
  \label{fig:c2d_matlab}
\end{figure}
which confirms the manual derivation in \Cref{eq:partial_frac}--\eqref{eq:zoh_numeric}.

\section{State Space Model}
Choosing the angular rate and linear position as states ($x = [\omega\; x_a]^T$) results in the continuous state equations
\begin{equation}
  \dot{x} = \begin{bmatrix}
    -B_e/J_e & 0\\
    K_e      & 0
  \end{bmatrix} x +
  \begin{bmatrix}
    K_a K_t/J_e\\[4pt]
    0
  \end{bmatrix} u,\qquad
  y = \begin{bmatrix}0 & 1\end{bmatrix} x.
\end{equation}
Discretizing with the ZOH at $T=0.0002~\si{s}$ gives
\begin{equation}
  x[k+1] = \underbrace{\begin{bmatrix}
    0.998252956 & 0\\
    6.3606\times10^{-4} & 1
  \end{bmatrix}}_{A_d} x[k] +
  \underbrace{\begin{bmatrix}
    0.182309\\[2pt]
    5.8048\times10^{-5}
  \end{bmatrix}}_{B_d} u[k],\qquad
  y[k] = \begin{bmatrix}0 & 1\end{bmatrix} x[k].
\end{equation}
The discrete state model and the transfer function in \Cref{eq:zoh_numeric} produce the same unit-step response, as shown in \Cref{fig:step_compare}.  Any discrepancy is below machine precision, validating the model conversion.

\begin{figure}[H]
  \centering
  \includegraphics[width=0.7\linewidth]{step_compare.png}
  \caption{Unit-step comparison between the discrete transfer function and discrete state-space model.}
  \label{fig:step_compare}
\end{figure}

\section{Prelab 3 -- Stability Analysis}
\subsection{Root Locus in $s$ and $z$ Domains}
The proportional position loop $K_p$ introduces a root locus that starts from the origin and the real mechanical pole.  The continuous locus in the left panel of \Cref{fig:rl_pair} shows that increasing $K_p$ pushes one pole deeper into the left half-plane while the other moves toward the right half-plane, crossing into instability once $K_p \approx 5.4~\si{V/mm}$.  The discrete root locus in the right panel mirrors this motion with critical crossings at $\lvert z\rvert=1$.

\begin{figure}[H]
  \centering
  \includegraphics[width=0.45\linewidth]{root_locus_continuous.png}\hfill
  \includegraphics[width=0.45\linewidth]{root_locus_discrete.png}
  \caption{Continuous (left) and discrete (right) root-locus plots for the proportional loop $K_p G_\text{ol}$.}
  \label{fig:rl_pair}
\end{figure}

\subsection{Gain and Phase Margins}
The open-loop frequency responses for the continuous and discrete models appear in \Cref{fig:bode_compare}.  Table~\ref{tab:margins} summarises the stability margins obtained from MATLAB's \texttt{margin} command (replicated with the Python Control Systems toolbox).  The continuous plant never crosses $0~\si{dB}$, so the gain margin is effectively infinite, but the phase margin is only $9.27^\circ$, signalling a lightly damped closed-loop response.  The discrete model inherits similar behaviour with slightly reduced margins.

\begin{figure}[H]
  \centering
  \includegraphics[width=0.75\linewidth]{bode_compare.png}
  \caption{Bode comparison between $G_\text{ol}(s)$ and $G_\text{ol}(z)$ with $T=0.2~\text{ms}$.}
  \label{fig:bode_compare}
\end{figure}

\begin{table}[H]
  \centering
  \caption{Gain and phase margins for different models (MATLAB \texttt{bode}/\texttt{margin}).}
  \label{tab:margins}
  \begin{tabular}{lcccc}
    \toprule
    Model & Gain Margin & Phase Margin & $\omega_{cg}$ [rad/s] & $\omega_{cp}$ [rad/s]\\
    \midrule
    $G_\text{ol}(s)$ & $>10^6$ (no crossing) & $9.27^\circ$ & -- & 53.54\\
    $G_\text{ol}(z)$, $T=0.2~\text{ms}$ & $30.1$ & $8.97^\circ$ & 295.64 & 53.54\\
    $G_\text{ol}(z)$, $T=2~\text{ms}$ & $3.02$ & $6.21^\circ$ & 93.37 & 53.52\\
    $G_\text{ol}(z)$, $T=20~\text{ms}$ & $0.31$ & $-20.3^\circ$ & 29.15 & 52.17\\
    \bottomrule
  \end{tabular}
\end{table}

\subsection{Sampling-Time Influence}
\Cref{fig:bode_sampling} overlays the discrete Bode magnitudes for three sampling times.  Coarser sampling ($T=20~\text{ms}$) amplifies high-frequency gain and erodes the phase margin, rendering the loop unstable even before gains are added.  Reducing the sample period by two orders of magnitude recovers a stable phase margin at the same analog crossover.  Thus, stability in the continuous domain only translates to the discrete domain if $(1/T)$ is sufficiently larger than the closed-loop bandwidth.

\begin{figure}[H]
  \centering
  \includegraphics[width=0.7\linewidth]{bode_sampling.png}
  \caption{Discrete Bode magnitudes versus sampling time.}
  \label{fig:bode_sampling}
\end{figure}

\subsection{Continuous vs.\ Discrete Stability}
Continuous and discrete stability are equivalent only when the sampling theorem is respected and sufficient phase is available at the commanded bandwidth.  The $T=0.2~\text{ms}$ design lies well below the mechanical time constant, so the poles remain inside the unit circle.  When $T$ increases, the effective phase lag introduced by the ZOH can exceed $90^\circ$, pushing the discrete poles outside the unit circle even though the analog poles remain in the left half-plane.

\section{Prelab 4 -- P-Controller Design}
The unity-gain crossover specification at $\omega=60~\si{rad/s}$ is met by solving
\begin{equation}
  1 = K_p \bigl\lvert G_\text{ol}(e^{j\omega T}) \bigr\rvert\big|_{\omega=60},
  \qquad\Rightarrow\qquad
  K_p = \frac{1}{0.7983} = 1.253~\si{V/mm}.
\end{equation}
The corresponding phase is $-172.1^\circ$, so the loop has a modest phase margin of $7.9^\circ$.  The nonlinear discrete simulation in \Cref{fig:pcontroller} applies this gain with Coulomb friction and amplifier saturation.  Increasing $\mu_k$ increases steady-state error and reduces overshoot, while tightening the current limits lengthens the rise and settling times due to current clipping.

\begin{figure}[H]
  \centering
  \includegraphics[width=0.85\linewidth]{pcontroller_nonlinear_effects.png}
  \caption{Effect of Coulomb friction (top) and amplifier saturation (bottom) on the digital P-controlled step response.}
  \label{fig:pcontroller}
\end{figure}

\section{Prelab 5 -- Lead--Lag Compensator}
\subsection{Lead Design at $\omega_c=377~\text{rad/s}$}
The open-loop phase at $\omega_c$ is $-178.67^\circ$, so only $1.33^\circ$ of phase margin remains.  To achieve $60^\circ$, an additional $\phi_{\max}=58.67^\circ$ must be supplied by the compensator.  Using the standard lead relations
\[
  \alpha = \frac{1+\sin\phi_{\max}}{1-\sin\phi_{\max}} = 12.717,\qquad
  \tau = \frac{1}{\omega_c\sqrt{\alpha}} = 7.44\times10^{-4}~\text{s},
\]
the compensator
\begin{equation}
  C_\text{LL}(s) = K\frac{\alpha \tau s + 1}{\tau s + 1}
  = 13.73\, \frac{0.1299\,s + 13.73}{0.0007438\,s + 1}
\end{equation}
provides the required magnitude at $\omega_c$.  \Cref{fig:lrr_lead} confirms that the loop return ratio now exhibits a $\qty{60}{\degree}$ phase margin at the commanded crossover.

\begin{figure}[H]
  \centering
  \includegraphics[width=0.75\linewidth]{lrr_lead.png}
  \caption{Loop return ratio after inserting the lead compensator; the $0~\text{dB}$ crossing occurs at $377~\si{rad/s}$ with $\qty{60}{\degree}$ phase margin.}
  \label{fig:lrr_lead}
\end{figure}

Discretizing $C_\text{LL}(s)$ with Tustin's method gives the implementable digital controller
\begin{equation}
  C_\text{LL}(z) = \frac{155.5 z - 152.3}{z - 0.763}, \qquad T = 0.2~\text{ms}.
\end{equation}

\subsection{Integral Augmentation}
The specified integral action is $C_I(s) = (s + K_i)/s$ with $K_i = \omega_c/10 = 37.7~\text{rad/s}$.  Cascading this block with $C_\text{LL}$ yields
\begin{equation}
  C_\text{LLI}(s) = C_\text{LL}(s) \frac{s + 37.7}{s}
  = \frac{0.1299 s^2 + 18.62 s + 517.5}{0.0007438 s^2 + s},
\end{equation}
and its discrete representation
\begin{equation}
  C_\text{LLI}(z) = \frac{156.1 z^2 - 307.8 z + 151.7}{z^2 - 1.763 z + 0.763}.
\end{equation}
The additional pole at the origin eliminates steady-state error caused by Coulomb friction.  To demonstrate this effect without Simulink, the disturbance torque $\mu_k=0.3~\si{Nm}$ was converted into an equivalent voltage disturbance of $0.47~\si{V}$ at the plant input.  \Cref{fig:lead_integrator} shows that the lead-only loop exhibits a steady-state offset under both step and ramp commands, while the integral action drives the error to zero at the cost of slower settling and reduced gain margin (Table~\ref{tab:lead_margins}).

\begin{figure}[H]
  \centering
  \includegraphics[width=0.8\linewidth]{lead_integral_tracking.png}
  \caption{Reference tracking in the presence of a constant friction disturbance: lead-only versus lead-plus-integral controllers.}
  \label{fig:lead_integrator}
\end{figure}

\begin{table}[H]
  \centering
  \caption{Loop stability margins after compensation.}
  \label{tab:lead_margins}
  \begin{tabular}{lcc}
    \toprule
    Loop & Gain Margin & Phase Margin\\
    \midrule
    $C_\text{LL} G_\text{ol}$ & $>10^6$ & $60.0^\circ$ at $377~\si{rad/s}$\\
    $C_\text{LLI} G_\text{ol}$ & $0.057$ (about $-25~\text{dB}$) & $54.3^\circ$ at $379~\si{rad/s}$\\
    \bottomrule
  \end{tabular}
\end{table}

\section{Discussion}
\Cref{fig:bode_summary} overlays the frequency responses of the plant, compensators, and compensated loops.  The lead network raises the magnitude near $\omega_c$ and injects positive phase, effectively shifting the crossover to the desired value.  Adding the integral factor increases low-frequency magnitude by \SI{20}{dB/dec}, guaranteeing zero steady-state error but reducing gain margin and introducing an additional $-90^\circ$ phase contribution.  In general:
\begin{itemize}
  \item Higher DC gain improves disturbance rejection and steady-state accuracy but can invite saturation and overshoot if left unchecked.
  \item Raising the gain crossover frequency shortens the rise time because the loop can track higher bandwidth commands, yet it also consumes phase margin.
  \item The coulomb friction plots in \Cref{fig:pcontroller} highlight that nonlinearities dominate overshoot and settling when actuator limits are tight, reinforcing the need for integral plus anti-windup logic in the full implementation.
\end{itemize}

\begin{figure}[H]
  \centering
  \includegraphics[width=0.9\linewidth]{bode_summary.png}
  \caption{Magnitude and phase of $G_\text{ol}(s)$, $C_\text{LL}(s)$, $C_\text{LLI}(s)$, and the combined open-loop transfer functions.}
  \label{fig:bode_summary}
\end{figure}

\clearpage
\bibliographystyle{ieeetr}
\begin{thebibliography}{1}
\bibitem{project_handout}
MECH467--541 Laboratory Team, ``Project II Handout: Ball Screw Feed Drive,'' 2025.
\end{thebibliography}

\end{document}
